\documentclass[10pt]{beamer}

\usetheme{metropolis}
\usepackage{fontspec}
\setsansfont[BoldFont={Fira Sans SemiBold}]{Fira Sans Book}
\setsansfont{Fira Sans}
\setmonofont{Fira Mono}
\usepackage{appendixnumberbeamer}

\usepackage{booktabs}
\usepackage[scale=2]{ccicons}

\usepackage{pgfplots}
\usepgfplotslibrary{dateplot}

\usepackage{xspace}
\newcommand{\themename}{\textbf{\textsc{metropolis}}\xspace}

\title{Metode Numerik dan Teknik Komputasi}
\subtitle{Integrasi Numerik}
\date{\today}
\author{Mifta Nur Farid, S.T., M.T.}
\institute{Teknik Elektro - Institut Teknologi Kalimantan \\ Karang Joang, Balikpapan}
\titlegraphic{\hfill\includegraphics[height=1.5cm]{OK-LOGO-ITK.jpg}}

\begin{document}

\maketitle

% \begin{frame}[fragile]{Pokok Bahasan}
%     \setbeamertemplate{section in toc}[sections numbered]    \tableofcontents
% \end{frame}

\section{Pendahuluan}

\begin{frame}[fragile]{Pendahuluan}
    \begin{itemize}[<+- | alert@+>]
        \item Penyelesaian lebih mudah untuk mencari nilai integral suatu fungsi yang cukup kompleks.
        \item Misal mencari integral pada $x=1,0$ hingga $x=2,8$ dari fungsi berikut
        $$f(x) = \frac{x^2 \ln(x) + e^{-x}}{5x \sin(x)}$$
        $$f(x) = \frac{x^2 \cos(x)}{e^{-x}}$$
        \item dan seterusnya.
    \end{itemize}
\end{frame}

\section{Metode - Metode Integral Numerik}

\begin{frame}[fragile]{Metode - Metode Integral Numerik}
    Metode - metode yang akan diajarkan pada Bab Integral Numerik adalah
    \begin{enumerate}[<+- | alert@+>]
        \item Trapezoida
        \item Simpson $\frac{1}{3}$
    \end{enumerate}
\end{frame}

\section{Trapezoida}

\begin{frame}[fragile]{Metode Trapezoida}
    \begin{itemize}[<+- | alert@+>]
        \item Metode mencari nilai integral dari fungsi $f(x)$ dengan batas tertentu (dari $x = x_0$ ke $x_n$).
        \item Kondisi \emph{non-equispaced}
        
        \begin{equation*}
            \int f(x) dx = \frac{(x_1 - x_0)}{2}(f_1 + f_0) + \cdots + \frac{(x_n - x_{n-1})}{2}(f_n + f_{n-1})
        \end{equation*}

        \item Kondisi \emph{equispaced}
        
        \begin{equation*}
            \int f(x) dx = \frac{h}{2}[f_0 + 2(f_1 + f_2 + \cdots +f_{n-1}) + f_n]
        \end{equation*}

        dimana $h = x_1 - x_0 = x_2 - x_1 = \cdots = x_n - x_{n-1}$
    \end{itemize}
\end{frame}

\begin{frame}[fragile]{Contoh Soal}
    \begin{columns}[T,onlytextwidth]
        \column{0.5\textwidth}
        \begin{table}
            \begin{tabular}{@{} ccc @{}}
            \toprule
            \textbf{n} &   \textbf{x} & \textbf{f(x)}\\
            \midrule
            0 &   1,0 & 1,449\\
            1 &   1,3 & 2,060\\
            2 &   1,6 & 2,645\\
            3 &   1,9 & 3,216\\
            4 &   2,2 & 3,779\\
            5 &   2,5 & 4,338\\
            6 &   2,6 & 4,898\\
            \bottomrule
            \end{tabular}
        \end{table}

        \column{0.5\textwidth}
        \begin{itemize}
            \item Carilah nilai integral dengan batas $x = 1,0$ hingga $x = 2,8$ dengan metode \textbf{Trapezoida}
        \end{itemize}
    \end{columns}
\end{frame}

\begin{frame}[fragile]{Contoh Soal}
    Karena merupakan tabel \alert{equispaced}, maka integral $f(x)$ dengan batas $x = 1,0$ hingga $x = 2,8$
    
    \begin{align*}
        \int f(x) dx =& \frac{h}{2}[f_0 + 2(f_1 + f_2 + f_3 + f_4 + f_5) + f_6]\\
        {} =& \frac{(1,3 - 1,0)}{2}[1,449 + 2(2,060 + 2,645 + 3,216 + 3,779 + 4,338)\\
        &+ 4,898] \\
        {} &= 5,76345
    \end{align*}
\end{frame}

\section{Simpson $\frac{1}{3}$}

\begin{frame}[fragile]{Metode Simpson $\frac{1}{3}$}
    \begin{itemize}[<+- | alert@+>]
        \item Metode mencari nilai integral fungsi $f(x)$ dengan batas tertentu (dari $x = x_0$ ke $x_n$)
        \item Hanya untuk kondisi equispaced
        $$\int f(x) dx = \frac{h}{3}[f_0 + 4(f_1 + f_3 + f_5 + \cdots + f_{n-1}) + 2(f_2 + f_4 + f_6 + \cdots + f_{n-2} + f_n]$$
        \item Lebih efektif jika n adalah genap
    \end{itemize}
\end{frame}

\begin{frame}[fragile]{Contoh Soal}
    \begin{columns}[T,onlytextwidth]
        \column{0.5\textwidth}
        \begin{table}
            \begin{tabular}{@{} ccc @{}}
            \toprule
            \textbf{n} &   \textbf{x} & \textbf{f(x)}\\
            \midrule
            0 &   1,0 & 1,449\\
            1 &   1,3 & 2,060\\
            2 &   1,6 & 2,645\\
            3 &   1,9 & 3,216\\
            4 &   2,2 & 3,779\\
            5 &   2,5 & 4,338\\
            6 &   2,6 & 4,898\\
            \bottomrule
            \end{tabular}
        \end{table}

        \column{0.5\textwidth}
        \begin{itemize}
            \item Carilah nilai integral dengan batas $x = 1,0$ hingga $x = 2,8$ dengan metode \textbf{Simpson $\frac{1}{3}$}
        \end{itemize}
    \end{columns}
\end{frame}

\begin{frame}[fragile]{Contoh Soal}
    Karena merupakan tabel \alert{equispaced}, maka integral $f(x)$ dengan batas $x = 1,0$ hingga $x = 2,8$
    
    \begin{align*}
        \int f(x) dx =& \frac{h}{3}[f_0 + 4(f_1 + f_3 + f_5) + 2(f_2 + f_4) + f_6]\\
        {} =& \frac{(1,3 - 1,0)}{3}[1,449 + 4(2,060 + 3,216 + 4,338) \\
        &+ 2(2,645 + 3,779) + 4,898 + 4,898] \\
        {} =& 5,7651
    \end{align*}
\end{frame}

\begin{frame}[standout]
    Ada Pertanyaan?
\end{frame}
  
\appendix
  
\end{document}